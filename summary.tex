\documentclass[11pt,a4paper, margin]{article}
\usepackage{amssymb,amsmath,amsthm,color,tabularx}
\usepackage{longtable,array}

\newtheoremstyle{break}
  {\topsep}{\topsep}%
  {\itshape}{}%
  {\bfseries}{}%
  {\newline}{}%
  
\theoremstyle{break}
\newtheorem*{problem}{Problem}

\theoremstyle{break}
\newtheorem*{solution}{Solution}

\newtheorem*{definition}{Def}
\newtheorem*{note}{Note}

\begin{document}

\title{MA-107 Summary}
\date{}
\maketitle

\section{Supply and Demand}

\section{Recurrence equations}
Equations relating values of $y$ at discrete points in time, expressing value $y_{t}$ of $y$ at time $t$ as a function of $y_{t-1}$, one unit of time before time $t$.

\subsection{Solving recurrence equations}
Solving recurrence equation refers to finding value of $y_{t}$ as a function of ${t}$.
			
\begin{problem}
Solve the recurrence equation \[y_{t} = ay_{t-1} + b\] for $t \geq 0$ given that $y_{0} = C$.
\end{problem}

\begin{solution}
After making sure that the equation you're solving is in the form \[y_{t} = ay_{t-1} + b\] you proceed as follows:
\begin{enumerate}
	\item If $a = 1$ your solution is $y_{t} = y_{0} + bt$. Otherwise:
	\item Make sure that your recurrence equation is of the form $y_{t} = ay_{t-1} + b$ for $a, b$ constants.
	\item Find $y^{\ast} = \dfrac{b}{1-a}$ where $a, b$ are constants from the equation above.
	\item Solution is $y(t) = y^{\ast} + (y_{0} - y^{\ast})a^{t}$
\end{enumerate}
\end{solution}

\subsection{Long term behaviour}
Describing behaviour of function $y_{t}$ as $t \rightarrow \infty$.

\begin{problem}
Given $ y_{t} = ay_{t-1} + b\ $ how does $y_{t}$ behave as $t$ tends to infinity?
\end{problem}

\begin{solution}
Assume recurrence equation $y_{t} = ay_{t-1} + b$ and time independent solution $y^{\ast}$, long term behaviour of $y_{t}$ \textbf{depends on coefficient} $\mathbf{a}$:
\begin{itemize}
	\item If $\mathbf{1 < a}$, then the function \textbf{increases unboundedly}($y_{t} \rightarrow +\infty$) if $y_{0} > y^{\ast}$ and the function \textbf{decreases unboundedly} $y_{t} \rightarrow -\infty$ $y_{0} < y^{\ast}$
	\item If $\mathbf{0 < a < 1}$, then the \textbf{function increases towards $y^{\ast}$}($y_{t} \rightarrow y^{\ast}$) if $y_{0} > y^{\ast}$ and the \textbf{function decreases towards $y^{\ast}$}($y_{t} \rightarrow y^{\ast}$) if $y_{0} < y^{\ast}$
	\item If $\mathbf{-1 < a < 0}$, then $y_{t}$ \textbf{oscillates towards $y^{\ast}$}.
	\item If $\mathbf{-1 < a < 0}$, then $y_{t}$ \textbf{oscillates unboundedly}.
\end{itemize}
\end{solution}

\subsection{Investment schemes}
Comparing return from different investments in terms of their present value.

\begin{problem}
Given access to a bank account with interest rate $r$ accrued annually at the end of each year and an asset with value function $V(t)$ calculate present value of selling said asset after $t$ years.
\end{problem}

\begin{solution}
Present value $P(t)$ is given by the equation \[ P(t) = V(t)(1+r)^{-t} \]
\end{solution}

\begin{problem}
Given access to a bank account with continuously compounded interest with rate $r$ and an asset with value function $V(t)$ calculate present value of selling said asset after $t$ years.
\end{problem}

\begin{solution}
Present value $P(t)$ is given by the equation \[ P(t) = V(t)e^{-tr} \]
\end{solution}

\begin{problem}
Given access to a bank account with annually compounded interest with rate $r$ calculate present value of:
\begin{enumerate}
	\item Receiving a sum $M$ after $n$ years.
	\item Receiving a yearly salary of $S$ for $n$ years.
	\item Selling an asset with value function $V(t)$ after $n$ years. 
\end{enumerate}
\end{problem}

\begin{solution}
Present value $P$ is given by the equation:
\begin{enumerate}
	\item $P(n) = M(1+r)^{-n}$
	\item $P(n) = S(r - \dfrac{r}{(1+r)^{n}})$
	\item $P(n) = V(n)(1+r)^{-n}$
\end{enumerate}
\end{solution}

\begin{problem}
Given access to a bank account with continuously compounded interest with rate $r$ calculate present value of:
\begin{enumerate}
	\item Receiving a sum $M$ after time $t$.
	\item Receiving a yearly salary of $S$ for $n$ years.
	\item Selling an asset with value function $V(t)$ after time $t$. 
\end{enumerate}
\end{problem}

\begin{solution}
Present value $P$ is given by the equation:
\begin{enumerate}
	\item $P(t) = Me^{-rt}$
	\item $P(n) = S(\dfrac{e^{rn}-1}{e^{rn}(e^r-1)})$
	\item $P(t) = V(t)e^{-tr}$
\end{enumerate}
\end{solution}

\section{Optimisation of functions in one variable}
Identifying stationary points (minima and maxima) and economically significant values such as marginal cost, break-even point, starting point etc.

\subsection{Differentiation}
Mathematical theory allowing for economical analysis of function's behaviour.

\begin{definition}[Lagrange's definition of a derivative]
Define the derivative of a function $f(x)$ with respect to $x$ to be 
\[f'(x) = lim_{h \rightarrow \infty} \dfrac{f(x+h) - f(x)}{h}\]
\end{definition}

Standard derivatives include:
\begin{itemize}
	\item $(e^x)'=e^x$
	\item $(ln(x))'=\dfrac{1}{x}$ for $x > 0$
	\item $(sin(x))'=cos(x)$
	\item $(cos(x))'=-sin(x)$
	\item $(x^n)'=nx^{n-1}$
\end{itemize}

\begin{definition}[Product rule]
For a product of two functions $f$ and $g$ we have that:
\[(f\cdot g)'=f'\cdot g + f\cdot g'\]
\end{definition}

\begin{definition}[Quotient rule]
For a quotient of two functions $f$ and $g$ we have that:
\[(\dfrac{f}{g})'=\dfrac{f'\cdot g + f\cdot g'}{g \cdot g}\]
\end{definition}

\begin{definition}[Chain rule]
For a composition of two functions $f$ and $g$ we have that:
\[(f \circ g)' = f'(g(x))g'(x)\]
\end{definition}

\begin{problem}
Describe nature of all stationary points of a function $f(x)$.
\end{problem}

\begin{solution}
\begin{enumerate}
	\item Calculate first derivative $f'(x)$ of $f$.
	\item Find all values of $x$ for which $f'(x)=0$. These are the \textbf{stationary points}.
	\item Compute value of second derivative $f''(p)$ at every previously found stationary point $p$.
	\begin{itemize}
		\item If $f''(p) > 0$ then $p$ is a (local) minimum.
		\item If $f''(p) < 0$ then $p$ is a (local) maximum.
		\item If $f''(p) = 0$ then look at the sign of a $f'(x)$ for $x$ on both sides if $p$.
		\begin{itemize}
			\item If $f'(x)$ changes sign from + to - at $p$ then we have a (local) maximum.
			\item If $f'(x)$ changes sign from - to + at $p$ then we have a (local) minimum.
			\item If sign of $f'(x)$ does not change sign at $p$ then $p$ is an inflection point (neither min nor max).
		\end{itemize}
	\end{itemize}
\end{enumerate}
\end{solution}

\subsection{Cost related optimisation}

\begin{definition} [Profit function]
Define \textbf{profit} $\Pi$ of a firm as a function of produced quantity $q$ to be:
\[\Pi(q) = pq - C(q)\]
where for a \textbf{small, efficient firm} we treat $p$ as a constant since the amount it produces does not affect market price.\\\\
Define indirect profit $\Pi^\ast$ of a firm to be a profit function $\Pi$ evaluated at its stationary point $q^\ast$ ie $\Pi^\ast(q) = pq^\ast - C(q^\ast)$. 
\end{definition}

\begin{definition} [Start-up and break-even points]

Given a supplier firm with cost and profit functions $C(q), \Pi(q)$ define:

\begin{itemize}
	\item \textbf{start-up point} $q_s$ to be the largest quantity $q$ produced that will bring the same profit as producing nothing. \\It can be found as a solution to:
\[\Pi(q_s) = \Pi(0)\]
	\item \textbf{break-even point} $q_b$ to be the smallest quantity $q$ produced that will cover the cost of production. \\It can be found as a solution to:
\[\Pi(q_b) = 0\]
\end{itemize} 
\end{definition}

\begin{problem} [Small firm]
Given a small efficient firm with cost function C(q):
\begin{itemize}
	\item Write the profit and find a value of selling price $p$ of the its product that maximises the profit for $q>0$.
	\item Find its start-up point $q_0$ and break-even point $q_b$. 
\end{itemize}
\end{problem}

\begin{solution} [Standard small firm procedure]
\begin{enumerate}
	\item Write profit function $\Pi(q)$ as defined above.
	\item Differentiate $\Pi(q)$ to show that it has a stationary point when $p = C'(q)$.
	\item Verify that it is a max by finding $\Pi''(q)$ for $p = C'(q)$.
	\item Find $q_0$ by solving $\Pi^\ast(q_0) = \Pi(0)$
	\item Find $q_b$ by solving $\Pi^\ast(q_b) = 0$
\end{enumerate}
\end{solution}

\section{Optimisation of functions in two variables}

\subsection{Differentiation}
Before when we had a function is one variable $f(x)$ we could write $f'(x)$ as its derivative with respect to $x$, namely $\dfrac{df(x)}{dx}$ or simply $\dfrac{df}{dx}$.\\ \\
With $f(x,y)$ being a function in two variables notation $f'(x,y)$ doesn't make much sense any more. So we define first order partial derivatives with respect to $x$ and $y$ to be \[f_x = \dfrac{\partial f}{\partial x} \quad \textrm{and} \quad f_y = \dfrac{\partial f}{\partial y} \quad \textrm{respectively}\] 
And second order derivatives are:
\[ f_{xx} = \dfrac{\partial^2f}{\partial x^2}, \quad f_{yy} = \dfrac{\partial^2f}{\partial y^2}, \quad f_{xy} = \dfrac{\partial^2f}{\partial x \partial y}, \quad f_{yx} = \dfrac{\partial^2f}{\partial y \partial x} \]

\begin{definition} [Chain rule]
Given function in two variables $F(x,y)$ and given that $x$ and $y$ are in turn functions of another variable $t$ we can define $f(t) = F(x(t), y(t))$ and get its derivative with respect to $t$ from chain rule as follows:
\[\dfrac{df}{dt} = \dfrac{\partial F}{\partial x}\dfrac{\partial x}{\partial t} + \dfrac{\partial F}{\partial y}\dfrac{\partial y}{\partial t}\] 
\end{definition}

\begin{problem}[Implicit differentiation]
Suppose we have a function $G(x,y)$ and that $y$ depends on $x$ and is being defined implicitly by $G(x,y) = C$. Find derivative of $y$ w.r.t. $x$ ie $\dfrac{dy}{dx}$
\end{problem}

\begin{solution}
Define $g(x) = G(x, y(x))$. From chain rule above we have that: \[\dfrac{dg}{dx} = \dfrac{\partial G}{\partial x} \dfrac{dx}{dx} + \dfrac{\partial G}{\partial y}\dfrac{dy}{dx}\]

From the implicit equation $G(x,y) = C$ we have that \[\dfrac{dg}{dx} = \dfrac{\partial G}{\partial x} = \dfrac{\partial C}{\partial x} = 0\]

Putting the two equations together we have that \[\dfrac{dg}{dx} = \dfrac{\partial G}{\partial x} \dfrac{dx}{dx} + \dfrac{\partial G}{\partial y}\dfrac{dy}{dx} = 0\]

Which after rearranging gives us: \[\dfrac{dy}{dx} = - \dfrac{\frac{\partial G}{\partial x}}{\frac{\partial G}{\partial y}} = - \dfrac{G_x}{G_y}\]

$G_x$ and $G_y$ can be easily computed by differentiating G w.r.t. $x$ and $y$
\end{solution}

\begin{definition}[Homogeneous function]
We say that a function $f(x,y)$ is homogeneous of degree $r$ if for any $\lambda$ we have: \[f(\lambda x, \lambda y) = \lambda^r f(x,y)\]
\end{definition}

\begin{definition} [Euler's theorem]
If $f(x,y)$ is homogeneous of degree $r$ then we have that:
\[x\dfrac{\partial f}{\partial x} + y\dfrac{\partial f}{\partial y} = rf(x,y)\]
\end{definition}

\end{document}